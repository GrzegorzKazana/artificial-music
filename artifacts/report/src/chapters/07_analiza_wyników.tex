\chapter{Analiza wyników}
{

    Jednym z napotkanych problemów w procesie rozwoju procesu przetwarzania danych i projektu architektury modelu
    był czasochłonny krok oceny wyników. Po każdorazowym uczeniu modelu konieczne było generowanie próbek, 
    konwersja ich postaci numerycznej na pliki midi, manualny odsłuch i subiektywna ocena.

    W celu formalizacji wyników i uzyskania możliwości wygodniejszego sposobu oceny modelu, konieczne było wyznaczenie
    miar opisujących generowane próbki. 

    \section{Miary i ich definicje}
    {
        %%% https://www.researchgate.net/publication/328728367_On_the_evaluation_of_generative_models_in_music
        Na podstawie artykułu [] zdecydowano się zaimplementować następujące miary:
        \begin{itemize}
            \item Rozpiętość tonalna - odległość między najniższym i najwyższym dźwiękiem, w przypadku plików midi ograniczona
            zakresem <0,127>,
            \item Histogram tonalny - rozkład występowania poszczególnych dźwięków z dwunastotonowej skali, niezależnie od oktawy, 
            \item Macierz przejść dźwięków - dwuwymiarowa macierz przedstawiająca częstotliwość występowania par dźwięków następujących
            po sobie. Uwagi wymaga kwestia nanoszenia na macierz informacji o następujących po sobie wielodźwiękach, w celu oddania
            również takich zdarzeń, na macierz nanoszone są pary będące wynikami iloczynu kartezjańskiego między składowymi wielodźwięków.
            \item Rozpiętość rytmiczna - odległość między najkrótszą i najdłuższą wartością rytmiczną występującą w mierzonej próbce. 
            Ponieważ wartości kodu 1 z N są posortowane według długości trwania, może być wyrażona poprzez różnicę indeksów 
            między klasami kodu 1 z N. Przyjmuje wartości z zakresu <0, ilość klas rytmicznych>.
            \item Histogram rytmiczny - rozkład występowania poszczególnych klas wartości rytmicznych.
            \item Macierz przejść wartości rytmicznych - opisuje częstość przejść między konkretnymi wartościami rytmicznymi.
        \end{itemize}

        Dodatkowo, opracowano dwie miary mierzące stopień kompresji generowanych ciągów dźwięków i wartości rytmicznych. 
        Intuicją idącą za uznaniem stopnia kompresji jako istotny parametr jest fakt wynikający z zasady działania dużej części
        algorytmów kompresji, w których powtarzające się ciągi znaków są zastępowane krótszym znacznikiem wskazującym na pierwsze 
        wystąpienie ciągu. 
        Oznacza to, że sekwencje charakteryzujące się większą powtarzalnością mogą zostać skompresowane w większym stopniu.
        Główną zaletą tej miary jest jej zdolność wykrywania powtarzających się ciągów odległych od siebie, co stanowi uzupełnienie 
        bardziej lokalnej miary, jaką jest macierz przejść.

        W celu możliwości skorzystania z dostępnych algorytmów, sekwencje dźwięków i wartości rytmicznych zostały przekształcone
        na ciągi znaków.

        Do kompresji wykorzystano bibliotekę zlib, korzystającą z algorytmu delfate.
    }

    \section{Wyniki}
    {

    }

    \section{Wrażenia subiektywne}
    {
        
    }
}