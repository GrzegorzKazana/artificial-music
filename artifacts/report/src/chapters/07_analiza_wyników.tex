\chapter{Analiza wyników}
{

    Jednym z napotkanych problemów w procesie rozwoju procesu przetwarzania danych i projektu architektury modelu
    był czasochłonny krok oceny wyników. Po każdorazowym uczeniu modelu konieczne było generowanie próbek, 
    konwersja ich postaci numerycznej na pliki midi, manualny odsłuch i subiektywna ocena.

    W celu formalizacji wyników i uzyskania możliwości wygodniejszego sposobu oceny modelu, konieczne było wyznaczenie
    miar opisujących generowane próbki. 

    \section{Miary i ich definicje}
    {
        %%% https://www.researchgate.net/publication/328728367_On_the_evaluation_of_generative_models_in_music
        Na podstawie artykułu [] zdecydowano się zaimplementować następujące miary:
        \begin{itemize}
            \item Rozpiętość tonalna - odległość między najniższym i najwyższym dźwiękiem, w przypadku plików midi ograniczona
            zakresem <0,127>
            \item Histogram tonalny - 
        \end{itemize}
    }

    \section{Wyniki}
    {

    }

    \section{Wrażenia subiektywne}
    {
        
    }
}