\chapter{Sposoby reprezentacji danych}
{
    Znaczna część modeli uczenia maszynowego, w tym sieci neuronowe, wymagają 
    danych w postaci numerycznej. Odpowiednio dobierając proces przekształcania danych
    jesteśmy w stanie nie tylko umożliwić skorzystanie z tych modeli, ale również możemy 
    uwypuklić informacje prawdziwie w nim istotne, co skróci i ułatwi proces uczenia modelu.

    Poniższy rozdział zawiera analizę różnych podejść do przedstawionego problemu.

    \section{Reprezentacja dźwięków}
    {
        Pierwszym z wymiarów naszych danych są wysokości dźwięków. W przypadku plików midi
        oryginalnie są to liczby z przedziału 0-127. Są to już dane numeryczne, które
        moglibyśmy teoretycznie wykorzystać jest bezpośrednio do uczenia modelu.
        W takim przypadku nasze zadanie sprowadziłoby się do problemu regresji, ponieważ próbowalibyśmy
        oszacować wartość numeryczną, którą następnie trzeba by ograniczyć do liczb z dozwolonego zakresu.

        Jedną z wad tego rozwiązania, jest fałszywe przedstawienie relacji między dźwiękami.
        Dźwięki znajdujące się w bliższym sąsiedztwie byłyby traktowane jako bardziej podobne.
        Takie założenie w muzyce nie zawsze jest prawdą. Przykładowo, dźwięki 60 i 66 (C4 i F\#4), znajdują się
        relatywnie blisko, mimo tego że występuje między nimi interwał trytonu, będący jednym z najsilniejszych dysonansów w skali 
        dwunastotonowej. Z drugiej strony, dźwięki 60 i 84 (C4 i C6) dzieli spora odległość, mimo tego że jest to ten sam dźwięk
        zagrany dwie oktawy wyżej.

        Niniejsze problemy można próbować zaadresować poprzez stosowanie poniższych podejść.

        \subsection{Kody 1 z N i M z N}
        {
            Klasycznym sposobem na rozwiązanie powyższego problemu jest zastosowanie kodu 1 z N. W przypadku
            informacji o dźwiękach midi, oznaczałoby to wypełnienie wektora 128 zerami, i postawieniem jedynki na 
            pozycji reprezentującej dany dźwięk. 
            Zaletą tego rozwiązania jest możliwość reprezentacji melodii polifonicznych, co przekształciłoby
            powyższy kod na kod M z N, gdzie M to ilość dźwięków granych jednocześnie w danym momencie sekwencji.
            Mimo tego, że w taki sposób pozbywany się fałszywych zależności między sąsiednimi dźwiękami, 
            to nie reprezentujemy zależności prawdziwych, jak w przypadku dźwięków odległych o oktawy. 
            Kolejnym problem tego podejścia jest również znaczący wzrost wymiarowości naszych danych i spadek ich kompresji,
            co przełoży się na wolniejszy proces uczenia.
        }

        \subsection{Wektory zanurzone}
        {
            Powyższy problem dużej wymiarowości i niemożliwości wyrażenia relacji między obiektami nie jest
            charakterystyczny tylko i wyłącznie dla analizy dźwięku. Problem ten występuje również w dziale
            przetwarzania języka naturalnego. Również w tym kontekście każdy wyraz stanowi
            jeden element z N, gdzie N jest rozmiarem słownika - zbioru wszystkich wyrazów.

            %%% link do word2veca
            Rozwiązaniem cieszącym się dużym powodzeniem jest algorytm Word2Vec zaproponowany w roku 2013 przez ... .
            Jest to algorytm służący do wyznaczenia ukrytej reprezentacji kodów 1 z N w postaci wektorów wartości ciągłych
            o dużo mniejszym wymiarze. Metoda opiera się na założeniu twierdzącym że wektory 
            o podobnym znaczeniu występują w sąsiedztwie podobnych lub nawet tych samych wyrazów. Ponieważ to założenie 
            jest również prawdziwe w kontekście muzyki przyjęto użycie algorytmy Word2Vec za podstawne.
            
            Jedyną modyfikacją potrzebną do wykorzystania tej metody do przykładów polifonicznych, było traktowanie 
            całych wielodźwięków jako pojedynczych obiektów w słowniku.
        }
    }

    \section{Reprezentacja czasu}
    {
        \subsection{Próbkowanie}
        {

        }

        \subsection{Czas jako zmienna ciągła}
        {

        }

        \subsection{Grupowanie długości dźwięków}
        {

        }
    }
}