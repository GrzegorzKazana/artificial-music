\chapter{Sposoby reprezentacji danych}
{
    Znaczna część modeli uczenia maszynowego, w tym sieci neuronowe, wymagają 
    danych w postaci numerycznej. Odpowiednio dobierając proces przekształcania danych
    jesteśmy w stanie nie tylko umożliwić skorzystanie z tych modeli, ale również możemy 
    uwypuklić informacje prawdziwie w nim istotne, co skróci i ułatwi proces uczenia modelu.

    Poniższy rozdział zawiera analizę różnych podejść do przedstawionego problemu.

    \section{Reprezentacja dźwięków}
    {
        Pierwszym z wymiarów naszych danych są wysokości dźwięków. W przypadku plików midi
        oryginalnie są to liczby z przedziału 0-127. Są to już dane numeryczne, które
        moglibyśmy teoretycznie wykorzystać jest bezpośrednio do uczenia modelu.
        W takim przypadku nasze zadanie sprowadziłoby się do problemu regresji, ponieważ próbowalibyśmy
        oszacować wartość numeryczną, którą następnie trzeba by ograniczyć do liczb z dozwolonego zakresu.

        Jedną z wad tego rozwiązania, jest fałszywe przedstawienie relacji między dźwiękami.
        Dźwięki znajdujące się w bliższym sąsiedztwie byłyby traktowane jako bardziej podobne.
        Takie założenie w muzyce nie zawsze jest prawdą. Przykładowo, dźwięki 60 i 66 (C4 i F\#4), znajdują się
        relatywnie blisko, mimo tego że występuje między nimi interwał trytonu, będący jednym z najsilniejszych dysonansów w skali 
        dwunastotonowej. Z drugiej strony, dźwięki 60 i 84 (C4 i C6) dzieli spora odległość, mimo tego że jest to ten sam dźwięk
        zagrany dwie oktawy wyżej.

        Niniejsze problemy można próbować zaadresować poprzez stosowanie poniższych podejść.

        \subsection{Kody 1 z N i M z N}
        {
            Klasycznym sposobem na rozwiązanie powyższego problemu jest zastosowanie kodu 1 z N. W przypadku
            informacji o dźwiękach midi, oznaczałoby to wypełnienie wektora 128 zerami, i postawieniem jedynki na 
            pozycji reprezentującej dany dźwięk. 
            Zaletą tego rozwiązania jest możliwość reprezentacji melodii polifonicznych, co przekształciłoby
            powyższy kod na kod M z N, gdzie M to ilość dźwięków granych jednocześnie w danym momencie sekwencji.
            Mimo tego, że w taki sposób pozbywany się fałszywych zależności między sąsiednimi dźwiękami, 
            to nie reprezentujemy zależności prawdziwych, jak w przypadku dźwięków odległych o oktawy. 
            Kolejnym problem tego podejścia jest również znaczący wzrost wymiarowości naszych danych i spadek ich kompresji,
            co przełoży się na wolniejszy proces uczenia.
        }

        \subsection{Wektory zanurzone}
        {
            Powyższy problem dużej wymiarowości i niemożliwości wyrażenia relacji między obiektami nie jest
            charakterystyczny tylko i wyłącznie dla analizy dźwięku. Problem ten występuje również w dziale
            przetwarzania języka naturalnego. Również w tym kontekście każdy wyraz stanowi
            jeden element z N, gdzie N jest rozmiarem słownika - zbioru wszystkich wyrazów.

            %%% link do word2veca
            Rozwiązaniem cieszącym się dużym powodzeniem jest algorytm Word2Vec zaproponowany w roku 2013 przez ... .
            Jest to algorytm służący do wyznaczenia ukrytej reprezentacji kodów 1 z N w postaci wektorów wartości ciągłych
            o dużo mniejszym wymiarze. Metoda opiera się na założeniu twierdzącym że wektory 
            o podobnym znaczeniu występują w sąsiedztwie podobnych lub nawet tych samych wyrazów. Ponieważ to założenie 
            jest również prawdziwe w kontekście muzyki przyjęto użycie algorytmy Word2Vec za podstawne.
            
            Jedyną modyfikacją potrzebną do wykorzystania tej metody do przykładów polifonicznych, było traktowanie 
            całych wielodźwięków jako pojedynczych obiektów w słowniku.
        }
    }

    \section{Reprezentacja czasu}
    {
        Drugim wymiarem danych muzycznych jest czas. Ponownie, wartość oznaczająca upływ czasu
        mogłaby użyta bezpośrednio, ponownie sprowadzając problem do regresji. Dodatkowo, rozsądnym krokiem
        byłoby znormalizowanie wartości w impulsach midi do arbitralnie wybranej wartości reprezentującą konkretną
        wartość rytmiczną. Przykładowym mapowaniem mogłoby być przyjęcie wartości 1 dla ćwierćnut.

        Głównym mankamentem tego rozwiązania jest spowodowane naturą regresji ryzyko niemożliwości wyuczenia 
        dokładnych wartości rytmicznych, a jedynie oscylowanie między wartościami. W takim przypadku generowana muzyka
        mogłaby nie sprawiać wrażenia rytmicznej, gdyż dźwięki nie tworzyłyby spójnych taktów. 

        \subsection{Próbkowanie}
        {
            Alternatywną metodą analizy upływu czasu jest próbkowanie. Z przyjętą częstotliwością próbkowania, 
            określałoby się listę aktualnie wybrzmiewających dźwięków i tworzyło macierz o wymiarach t x n, gdzie n to
            wymiarowość reprezentacji wysokości dźwięku, a t to ilość pobranych próbek w danym fragmencie utworu.
            
            Ponieważ dźwięki zawsze trwają wielokrotność pewnej wartości, potencjalnie łatwiejsze byłoby 
            odtworzenie rytmicznego charakteru muzyki.
            Wadą tego podejścia jest konieczność doboru częstotliwości próbkowania. Zbyt niskie wartości 
            uniemożliwiłyby precyzyjne przestawienie utworu. Natomiast, wraz z wyborem większej częstotliwości próbkowania,
            te same dźwięki byłyby reprezentowane przez większą ilość wektorów, 
            co powodowały oddalenie następujących dźwięków w sekwencji i utrudnienie uczenia.
        }

        \subsection{Grupowanie długości dźwięków}
        {
            Podejściem łączącym zalety obydwu rozwiązań, jest przeprowadzenie grupowania znormalizowanych czasów w
            odniesieniu do ustalonej wartości rytmicznej. W ten sposób możliwe byłoby odzwierciedlenie wartości rytmicznych
            występujących w zbiorze danych i przedstawienie ich w dyskretny sposób bez arbitralnego określania zbioru dozwolonych
            wartości. 
            Ponownie, po podziale na grupy problem zostanie przekształcony do problemu klasyfikacji 1 z N.
        }
    }

    \section{Podsumowanie}
    {
        Z powodu zalet i wad przedstawionych w powyższej analizie, zdecydowano się na reprezentacje dźwięków w postaci
        wektorów zanurzonych, a wybraną reprezentacją czasu zostały dyskretne wartości uzyskane poprzez grupowanie.
    }
}