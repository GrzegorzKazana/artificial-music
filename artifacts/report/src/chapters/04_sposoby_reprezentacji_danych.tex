\chapter{Sposoby reprezentacji danych}
{
    Znaczna część modeli uczenia maszynowego, w\,\,tym sieci neuronowe, wymagają 
    danych w\,\,postaci numerycznej. Odpowiednio dobierając proces przekształcania danych
    możliwe jest nie tylko dostosowanie danych do metody uczenia, ale również
    uwypuklenie informacji prawdziwie w\,\,nich istotnych, co potencjalnie skróci i\,\,ułatwi proces uczenia modelu.

    Poniższy rozdział zawiera analizę różnych podejść do przedstawionego problemu.

    \section{Reprezentacja wysokości dźwięków}
    {
        Pierwszym z\,\,wymiarów danych muzycznych są wysokości dźwięków. W\,\,przypadku plików midi
        oryginalnie są to liczby z\,\,przedziału 0-127. Są to już dane numeryczne, które
        można by teoretycznie wykorzystać bezpośrednio do uczenia modelu.
        W\,\,takim przypadku zadanie sprowadziłoby się do problemu regresji, ponieważ model próbowałby
        oszacować wartość numeryczną, którą następnie trzeba by ograniczyć do liczb całkowitych z\,\,dozwolonego zakresu.

        Jedną z\,\,wad tego rozwiązania, jest przedstawienie fałszywych relacji między dźwiękami.
        Dźwięki znajdujące się w\,\,bliższym sąsiedztwie byłyby traktowane jako bardziej sobie podobne.
        Takie założenie w\,\,muzyce nie zawsze jest prawdą. Przykładowo, dźwięki 60 i\,\,66 (C4 i\,\,F\#4), znajdują się
        relatywnie blisko, mimo tego że występuje między nimi interwał trytonu, będący jednym z\,\,najsilniejszych dysonansów w\,\,skali 
        dwunastotonowej. Z\,\,drugiej strony, dźwięki 60 i\,\,84 (C4 i\,\,C6) dzieli spora odległość, mimo tego że jest to ten sam dźwięk
        zagrany dwie oktawy wyżej.

        Z\,\,niniejszymi problemami można się mierzyć poprzez stosowanie poniższych podejść.

        \subsection{Kody 1 z N i M z N}
        {
            Klasycznym sposobem na rozwiązanie powyższego problemu jest zastosowanie kodu 1 z\,\,N. W\,\,przypadku
            informacji o\,\,dźwiękach midi, oznaczałoby to wypełnienie wektora 128 zerami i\,\,postawienie jedynki na 
            pozycji reprezentującej dany dźwięk. 
            Zaletą tego rozwiązania jest możliwość reprezentacji melodii polifonicznych, co przekształciłoby
            powyższy kod w\,\,kod M\,\,z N, gdzie M\,\,to ilość dźwięków wybrzmiewających jednocześnie w\,\,danym momencie sekwencji.
            Mimo tego, że w\,\,taki sposób można uniknąć problemu fałszywych zależności między sąsiednimi dźwiękami, 
            to nie są reprezentowane również zależności mogących być użytecznymi w\,\,procesie uczenia, 
            takich jak w\,\,przypadku reprezentacji dźwięków odległych o\,\,oktawy. 
            Kolejnym problemem tego podejścia jest również znaczący wzrost wymiarowości danych i\,\,spadek ich kompresji,
            co przełoży się na wolniejszy proces uczenia.
        }

        \subsection{Wektory zanurzone}
        {
            Problem dużej wymiarowości i\,\,niemożliwości wyrażenia relacji między obiektami nie jest
            charakterystyczny tylko i\,\,wyłącznie dla analizy dźwięku. Zagadnienie to  występuje również w\,\,dziale
            przetwarzania języka naturalnego. Również w\,\,kontekście modeli operujących na fragmentach języka każdy wyraz stanowi
            jeden element z\,\,N, gdzie N\,\,jest rozmiarem słownika - zbioru wszystkich wyrazów.

            Rozwiązaniem cieszącym się dużą popularnością jest algorytm Word2Vec zaproponowany w\,\,roku 2013 \cite{Mikolov2013EfficientEO}.
            Jest to algorytm służący do wyznaczenia ukrytej reprezentacji kodów 1 z\,\,N w\,\,postaci wektorów wartości ciągłych
            o\,\,dużo mniejszym wymiarze. Metoda opiera się na założeniu twierdzącym, że wektory 
            o\,\,podobnym znaczeniu występują w\,\,sąsiedztwie podobnych lub nawet tych samych wyrazów. Ponieważ to założenie 
            jest również prawdziwe w\,\,kontekście muzyki przyjęto użycie algorytmu Word2Vec za podstawne.
            
            Jedyną modyfikacją potrzebną do wykorzystania tej metody do przykładów polifonicznych, było traktowanie 
            całych wielodźwięków jako pojedynczych obiektów w\,\,słowniku. Pozwoliła na to transformacja wektorów kodu 
            M\,\,z N\,\,na ciągi znaków reprezentujących wybrzmiewające dźwięki.
        }
    }

    \section{Reprezentacja czasu}
    {
        Drugim wymiarem danych muzycznych jest czas. Ponownie, wartość oznaczająca upływ czasu
        mogłaby być użyta bezpośrednio, znów sprowadzając problem do regresji. Dodatkowo, rozsądnym krokiem
        byłoby znormalizowanie wartości w\,\,impulsach midi do arbitralnie wybranej wartości reprezentującej konkretną
        wartość rytmiczną. Przykładowym mapowaniem mogłoby być przyjęcie wartości 1 dla ćwierćnut.

        Głównym mankamentem tego rozwiązania, jest spowodowane naturą regresji ryzyko niemożliwości precyzyjnego wyuczenia 
        dokładnych wartości rytmicznych, a\,\,jedynie oscylowanie między wartościami. W\,\,takim przypadku generowana muzyka
        mogłaby sprawiać wrażenia arytmicznej, gdyż następujące dźwięki nie tworzyłyby spójnych taktów. 

        \subsection{Próbkowanie}
        {
            Alternatywną metodą wyrażenia upływu czasu jest próbkowanie. Z\,\,przyjętą częstotliwością próbkowania, 
            możliwe byłoby tworzenie listy aktualnie wybrzmiewających dźwięków i\,\,zapełnienie macierzy o\,\,wymiarach \(t \times n\), gdzie n\,\,to
            wymiarowość reprezentacji wysokości dźwięku, a\,\,t to ilość pobranych próbek w\,\,danym fragmencie utworu.
            
            Ponieważ dźwięki zawsze trwają wielokrotność pewnej wartości, można twierdzić, że łatwiejsze byłoby 
            odtworzenie rytmicznego charakteru muzyki.
            
            Wadą tego podejścia jest konieczność doboru częstotliwości próbkowania. Zbyt niskie wartości 
            uniemożliwiłyby precyzyjne przestawienie utworu. Natomiast, wraz z\,\,wyborem większej częstotliwości próbkowania,
            te same dźwięki byłyby reprezentowane przez większą ilość wektorów, 
            co powodowały oddalenie następujących po sobie dźwięków w\,\,sekwencji i\,\,utrudnienie uczenia.
        }

        \subsection{Grupowanie długości dźwięków}
        {
            Podejściem łączącym zalety obydwu rozwiązań, jest przeprowadzenie grupowania znormalizowanych wartości parametru czasu w
            odniesieniu do ustalonej wartości rytmicznej. W\,\,ten sposób możliwe byłoby odzwierciedlenie wartości rytmicznych
            występujących w\,\,zbiorze danych i\,\,przedstawienie ich w\,\,dyskretny sposób bez arbitralnego określania zbioru dozwolonych
            wartości. 
            Po podziale wartości rytmicznych na niezależne klasy problem zostanie przekształcony do problemu klasyfikacji 1\,z\,N. 

            W celu otrzymania grup odpowiadających najczęstszym wartościom rytmicznym można skorzystać 
            z\,\,algorytmów klasteryzujących takich jak k-Means \cite{MacQueen1967SomeMF} lub DBSCAN \cite{Ester1996ADA}.
        }
    }

    \section{Podsumowanie}
    {
        Z\,\,powodu zalet i\,\,wad każdej z\,\,metod przedstawionych w\,\,powyższej analizie, zdecydowano się na reprezentacje dźwięków w\,\,postaci
        wektorów zanurzonych, a\,\,wybraną reprezentacją czasu zostały dyskretne wartości uzyskane poprzez grupowanie.
    }
}