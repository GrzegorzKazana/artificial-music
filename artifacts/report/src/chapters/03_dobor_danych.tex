\chapter {Dobór danych} 
{
    Jednym z kluczowych aspektów każdego procesu analizy danych i uczenia maszynowego
    jest odpowiedni dobór danych. Decyzje podjęte na tym etapie pociągają za sobą 
    konsekwencje w kolejnych etapach procesu. W zależności od postaci danych wejściowych,
    określany jest konieczny nakład pracy w procesie ich obróbki. 
    Ponadto, odpowiedni dobór danych ma znaczący wpływ na otrzymywane wyniki.

    W przypadku danych muzycznych należy zwrócić między innymi na cechy takie jak:
    \begin{itemize}
        \item dostępność danych treningowych,
        \item sposób reprezentacji upływu czasu,
        \item sposób wyrażenia wysokości dźwięku,
        \item jasne przedstawienie zależności między dźwiękami,
        \item stopień kompresji,
        \item złożoność samego formatu oraz możliwość istnienia niepoprawnych reprezentacji.
    \end{itemize}

    \section{Porównanie formatów danych}
    {
        \subsection{Pliki audio}
        {
            Najmniej restrykcyjnymi formatami plików audio są formaty będące zapisem
            amplitudy fali akustycznej. Przykładami plików tego rodzaju są pliki WAV
            oraz AIFF.
            Kluczowym parametrem plików audio jest częstotliwość próbkowania. Od niej
            zależy zakres możliwych do wyrażenia częstotliwości. Typową wartością dla plików
            %%% link do twierdzenia ???
            WAV jest 44100Hz, co według twierdzenia Nyquista-Shannona przekłada się na zakres
            częstotliwości ograniczony wartością 22050Hz, będącą bliską granicy słyszalności
            ucha ludzkiego. 

            Ponieważ wysokości dźwięków wyrażone są jedynie poprzez sekwencje zmian amplitudy sygnału,
            ich ekstrakcja wymagałaby wykorzystania transformaty Fouriera.

            Największą zaletą i wadą tego formatu jednocześnie jest jego swoboda. 
            Postać fali akustycznej pozwala na przedstawienie każdego dźwięku będącego w paśmie
            przenoszenia, za równo złożonych wieloinstrumentowych kompozycji muzycznych, 
            prostych melodii, jak i dźwięków nie podlegających pod definicję muzyki. 
            Prowadzi to do niskiego stopnia kompresji informacji, wymuszającego złożone metody obróbki i skomplikowane
            modele uczenia.
        }

        \subsection{Notacja ABC}
        {
            Notacja ABC jest jedną z cyfrowych postaci klasycznego zapisu nutowego.
            Dane w plikach ABC przedstawione są za pomocą znaków ASCII.

            Ponieważ jest to odpowiednik zapisu nutowego, oznacza to
            że za równo wysokości dźwięków, jak i wartości rytmiczne należą do dyskretnego zbioru.
            Fakt ten nakłada spore ograniczenie na treść plików, lecz w kontekście muzyki jest to 
            ograniczenia bardzo pomocne.

            Poza informacją o samym dźwięku i jego wartości rytmicznej, format zawiera również metadane
            utworu takie jak autor, sygnatura i tempo.

            %%% jakiś przykładowy obrazek z zapisem

            Dużą zaletą tego formatu jest jego wysoki stopień kompresji, za pomocą relatywnie krótkich ciągów
            znaków jesteśmy w stanie opisać znaczną część utworu muzycznego. Głównym mankamentem notacji ABC
            w kontekście niniejszej pracy jest fakt, że nie wszystkie ciągi znaków są poprawnymi zapisami 
            w notacji ABC. Oznacza to, że pojedyncze błędy w generowanych sekwencjach mogą powodować niepoprawność całego utworu.
            W przypadku niemożliwości wyuczenia modelu tworzenia całkowicie poprawnych w ciągów wymagałoby
            walidacji i obsługi poprawiania błędów w zapisie.
        }

        \subsection{Format midi}
        {
            Pliki w formacie midi są zapisem wiadomości przesyłanych protokołem komunikacyjnym o ten samej nazwie.
            Każdy plik midi składa się z kanałów oraz ścieżek. Poszczególne kanały reprezentują poszczególne instrumenty
            biorące udział w nagraniu. Na poszczególnych ścieżkach znajdują się wiadomości reprezentujące zdarzenia w utworze.
            Do najczęstszych z nich należą: note\textunderscore on, note\textunderscore off, set\textunderscore tempo.

            W przypadku wiadomości związanych z rozpoczęciem lub zakończeniem
            dźwięku dołączony jest również numer dźwięku z zakresu 0-127. 
            Wartości tego parametru odpowiadają kolejnym dźwiękom skali dwunastotonowej poczynając od C-1 i kończąc na G9.
            Dostępne dźwięki są nadzbiorem dźwięków dostępnych na klawiaturze klasycznego fortepianu.

            Dodatkowym parametrem każdej wiadomości jest wartość time, będąca ilością ticków które upłynęły
            od poprzedniej wiadomości. Oznacza to, że długość trwania dźwięku można wyznaczyć poprzez zsumowanie parametrów time
            wiadomości występujących pomiędzy odpowiadającymi sobie zdarzeniami note\textunderscore on i note\textunderscore off.

            Możliwa jest konwersja wartości za równo na czas w milisekundach jak i wartości rytmiczne.
            Stałe potrzebne do tych przekształceń są naczęściej zawarte w meta wiadomościach znajdujących się na początku utworu.

            Zaletą formatu midi jest dyskretna reprezentacja wysokości dźwięku oraz pozwalająca na elastyczną 
            interpretację postać upływu czasu.
        }

        \subsection{Podsumowanie}
        {
            Wszystkie omówione formaty danych mają wady i zalety. Cechą opisującą wszystkie z nich, jest 
            bogata dostępność danych, przez co wybór dokonano na podstawie pozostałych cech.
            W dalszej części pracy opisywane podejścia będą tyczyć jedynie plików w formacie midi, na którego 
            wybór zdecydowano się drogą eliminacji. Pliki audio odrzucono przez wymaganą złożoność wstępnej obróbki,
            a notację ABC z powodu obaw związanych z trudnościami zagwarantowania syntaktycznej poprawności 
            generowanych ciągów znaków.
        }
    }

    \section{Opis wybranego zbioru danych}
    {

    }
}