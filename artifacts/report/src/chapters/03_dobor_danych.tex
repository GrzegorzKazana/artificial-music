\chapter {Dobór danych} 
{
    Jednym z kluczowych aspektów każdego procesu analizy danych i uczenia maszynowego
    jest odpowiedni dobór danych. Decyzje podjęte na tym etapie pociągają za sobą 
    konsekwencje w kolejnych etapach procesu. W zależności od postaci danych wejściowych,
    określany jest konieczny nakład pracy w procesie ich obróbki. 
    Ponadto, odpowiedni dobór danych ma kolosalny wpływ na otrzymywane wyniki.

    W przypadku danych muzycznych należy zwrócić między innymi na cechy takie jak:
    \begin{itemize}
        \item sposób reprezentacji upływu czasu,
        \item sposób wyrażenia wysokości dźwięku,
        \item jasne przedstawienie zależności między dźwiękami,
        \item stopień kompresji,
        \item złożoność samego formatu oraz możliwość istnienia niepoprawnych reprezentacji.
    \end{itemize}

    \section{Porównanie formatów danych}
    {
        \subsection{Pliki audio}
        {
            Najmniej restrykcyjnymi formatami plików audio są formaty będące zapisem
            amplitudy fali akustycznej. Przykładami plików tego rodzaju są pliki WAV
            oraz AIFF.
            Kluczowym parametrem plików audio jest częstotliwość próbkowania. Od niej
            zależy zakres możliwych do wyrażenia częstotliwości. Typową wartością dla plików
            %%% link do twierdzenia ???
            WAV jest 44100Hz, co według twierdzenia Nyquista-Shannona przekłada się na zakres
            częstotliwości ograniczony wartością 22050Hz, będącą bliską granicy słyszalności
            ucha ludzkiego. 

            Ponieważ wysokości dźwięków wyrażone są jedynie poprzez sekwencje zmian amplitudy sygnału,
            ich ekstrakcja wymagałaby wykorzystania transformaty Fouriera.

            Największą zaletą i wadą tego formatu jednocześnie jest jego swoboda. 
            Postać fali akustycznej pozwala na przedstawienie każdego dźwięku będącego w paśmie
            przenoszenia, za równo złożonych wieloinstrumentowych kompozycji muzycznych, 
            prostych melodii, jak i dźwięków nie podlegających pod definicję muzyki. 
            Prowadzi to do niskiego stopnia kompresji informacji, wymuszającego złożone metody obróbki i skomplikowane
            modele uczenia.

        }

        \subsection{Notacja ABC}
        {
            
        }

        \subsection{Format midi}
        {
            
        }

        \subsection{Podsumowanie}
        {
            
        }
    }

    \section{Opis wybranego zbioru danych}
    {

    }
}