\chapter{Wnioski}
{
    W\,\,niniejszej pracy przedstawiono jedno z\,\,możliwych podejść do rozwiązania problemu,
    jakim jest generowanie muzyki za pomocą metod uczenia maszynowego. 

    Udało się z powodzeniem spełnić podstawowe cele pracy, do których należał dobór i obróbka danych, 
    przeprowadzenie procesu uczenia maszynowego i ocena wyników. Pomimo nie osiągnięcia w\,\,pełni satysfakcjonujących rezultatów, cały proces był niezwykle pouczający. 
    Liczność oraz daty powstania dostępnych źródeł literaturowych wskazują, że problem jest aktualny 
    i\,\,wciąż prowadzone są badania na temat zastosowania metod sztucznej inteligencji w\,\,zadaniach, 
    które można określić mianem twórczych. Bogata treść studiowanych prac przedstawia szeroką gamę 
    różnych podejść do powyższego problemu.

    Kształcącym elementem pracy był również rozległy proces wstępnej analizy możliwych sposobów
    rozwoju projektu. Do etapów składających się na finalne efekty pracy należało porównanie odmiennych 
    formatów danych muzycznych, opracowanie numerycznej reprezentacji danych, dobór zbioru treningowego
    oraz samego projektu architektury modelu. 
    Razem z\,\,opracowywaniem metod przetwarzania danych będących intuicyjnie efektywniejszymi, tworzone również były założenia
    ograniczające wyniki i\,\,potencjalne zastosowania. Przykładem takiego kompromisu jest automatyczne wykrywanie
    najczęstszych wartości rytmicznych, pozwalające na dyskretyzację reprezentacji czasowej bez
    manualnego definiowania zbioru dozwolonych wartości, odbywające się kosztem całkowitej swobody jaką oferuje format midi.
    
    Do ograniczeń opisanego procesu można zaliczyć między innymi niemożliwość reprezentowania utworów wieloinstrumentowych, 
    ograniczony słownik wielodźwięków oraz stratność przekształcenia plików midi na macierze i\,\,ponownego przetworzenia
    macierzy na plik midi.

    W\,\,rozdziale dotyczącym uczenia modelu zaprezentowano podstawowe kroki, jakie należy poczynić podchodząc do
    dowolnego problemu uczenia maszynowego. Do najważniejszych z\,\,nich należy projekt architektury modelu,
    dobór miar i\,\,funkcji kosztów oraz walidacja modelu. Opisywany etap pracy również pozwolił na zapoznanie 
    się z\,\,współczesnymi narzędziami ułatwiającymi wszelkie zadania uczenia maszynowego, takimi jak Tensorflow, 
    TensorBoard i\,\,Google Colab. 

    \bigskip

    Otrzymane rezultaty nie spełniły wszystkich oczekiwań, lecz wykazywały conajmniej poprawność i\,\,częściowy
    sukces obranego podejścia. Jedną z\,\,przyczyn zastrzeżeń mógł być potencjalnie sam zbiór danych, nie 
    zawierający wystarczającej różnorodności pozwalającej na skuteczną generalizację zależności opisujących 
    powszechnie akceptowalne zasady harmonii i\,\,rytmiki.

    W\,\,celu osiągnięcia rezultatów spełniających bardziej rygorystyczne oczekiwania, słusznym krokiem byłoby
    ponowne wybranie zbioru danych treningowych, tym razem składającego się z\,\,większej ilości próbek i\,\,nieobarczonego
    wadą opisaną w\,\,rozdziale poświęconym procesowi uczenia. 
    Dodatkowo, słusznym krokiem mogłoby okazać się użycie metod augmentacji danych. W\,\,przypadku muzyki, mogłoby to
    być transponowanie utworów na inne tonacje, co wymusiłoby na modelu interpretacje zależności między interwałami
    muzycznymi, a\,\,nie konkretnymi dźwiękami.
    Wraz ze zmianą danych oraz ich objętości, konieczna byłaby ponowna parametryzacja modelu i\,\,procesu uczenia.

    \bigskip

    Interesującym kierunkiem, który jest podatny na dalszą eksplorację są inne rodzaje architektur sieci neuronowych.
    Pomimo że problem inherentnie opiera się na analizie sekwencji, co silnie sugeruje wykorzystanie rekurencyjnych 
    sieci typu Long Short-term Memory, możliwe są konstrukcje modeli odmienne od przytoczonych. Jedną z\,\,nich są modele typu 
    sequence-to-sequence encode-decoder, na których wyjściu i\,\,wejściu są sieci LSTM, lecz dane pomiędzy nimi są
    kompresowane do jednowymiarowego tensora, który następnie jest ponownie przekształcany na sekwencję. 
    Alternatywnym podejściem szeroko stosowanym w\,\,problemach transferu stylu i\,\,generacji obrazów, 
    są modele generacyjno-adwersyjne (Generative-Adversial Models), składające się sieci uczącej się generowania 
    próbek i\,\,sieci uczącej się rozróżniać oryginały od falsyfikatów.

    %%%
    \bigskip

    Generowanie utworów muzycznych jest jednym z\,\,problemów przekraczających granice, które
    jeszcze w\,\,nieodległej przeszłości wydawały się niemożliwe do przekroczenia. Zadanie to nakierowuje na wiele pytań 
    dotyczących definicji inteligencji, kreatywności oraz zdolności sztucznej inteligencji do tworzenia sztuki. 
    W\,\,niedalekiej przyszłości można spodziewać się stosowania metod uczenia maszynowego na szeroką skalę również w\,\,dziedzinach,
    które obecnie uznaje się za wyłącznie osiągalne przez człowieka. 

}
