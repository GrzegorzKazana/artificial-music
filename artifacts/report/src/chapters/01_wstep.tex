
\chapter{Wstęp}
{

%  \begin{itemize}
%      \item wprowadzenie w problem / zagadnienie
%      \item osadzenie problemu w dziedzinie
%      \item cel pracy
%      \item zakres pracy
%      \item zwięzła charakterystyka rozdziałów
%      \item jednoznacznie określenie wkładu autora

%  \end{itemize}


    Współczesne zastosowania metod uczenia maszynowego są bardzo szerokie.
    Z pośród niezliczonej ilości problemów praktycznie niemożliwych do rozwiązania podejściami klasycznymi, 
    duża część zostaje rozwiązana z wykorzystaniem szerokiej gamy algorytmów oraz modeli uczenia.
        
    Zadania stawiane przed systemami sztucznej inteligencji można podzielić między innymi na problemy:
    \begin{itemize}
        \item klasyfikacji - polegające na przypisaniu przykładu do określonej kategorii, 
        np. rozpoznawanie obiektów na obrazie 
        \item grupowania - polegające na wyodrębnieniu przypadków o podobnych cechach 
        lub współwystępujących, np. systemy sugerujące przedmioty w sklepach internetowych
        \item regresji - polegające na oszacowaniu wartości ciągłej dla poszczególnych przykładów, 
        np. przewidywanie cen produktów.
    \end{itemize}
    Pomimo doniosłości oraz znaczenia zastosowań powyższych zagadnień,
    nie można w nich dostrzec przejawów kreatywności.
    Jednym z problemów, w którym można doszukiwać się inwencji jest tworzenie muzyki.

    \section{Cel i zastosowania}
    {
        Celem pracy jest analiza różnych podejść do obróbki plików muzycznych oraz 
        porównanie wyników sposobów syntezy próbek z wykorzystaniem metod uczenia maszynowego. 

        Pomimo że praca ma charakter badawczo-eksperymentatorski, 
        opracowane metody oraz ich wyniki mogą potencjalnie znaleźć zastosowanie
        w zadaniach ekstrakcji stylu, augmentacji istniejących zbiorów danych,
        lub w optymistycznym scenariuszu jako narzędzie wspierające proces twórczy muzyków.
    }

    \section{Zakres pracy}
    {
        Do zadań autora pracy należy
        \begin{itemize}
            \item dobór zbioru danych
            \item dobór, implementacja i analiza metod przetwarzania plików muzycznych
            \item dobór i parametryzacja architektury modelu uczenia maszynowego
            \item wykorzystanie przygotowanych danych 
            w procesie uczenia i syntezy nowych próbek
            \item ocena uzyskanych wyników
        \end{itemize}
    }


    % Muzyką nazywamy ciągi dźwięków tworzące kompozycyjną całość. 
    % Utwory muzyczne można analizować pod wieloma względami, takimi jak:
    % \begin{itemize}
    %     \item rytmiczność - organizacja dźwięków w czasie
    %     \item melodyczność - sposób zestawiania następujących dźwięków 
    %     \item harmoniczność - spójność i ład występujący między dźwiękami
    %     \item dynamika - zróżnicowanie siły dźwięków
    % \end{itemize}

    % \section{Osadzenie problemu w dziedzinie}
    % {
    %     Generowanie muzyki można rozumieć jako zadanie polegające 
    %     na ekstrakcji pewnych cech charakterystycznych
    %     przykładowych utworów, i wykorzystaniu ich przy syntezie tworzonych próbek. 
    % }

    % \section{Cel pracy}
    % {
    %     Celem pracy jest analiza i porównanie wyników różnych podejść 
    %     do obróbki plików muzycznych oraz sposobów syntezy próbek 
    %     z wykorzystaniem metod uczenia maszynowego. 
    % }

    % \section{Zakres pracy i wkład autora}
    % {
    %     Do zadań autora pracy należy
    %     \begin{itemize}
    %         \item dobór, implementacja, analiza metod przetwarzania plików muzycznych
    %         \item dobór i parametryzacja architektury modelu uczenia maszynowego
    %         \item wykorzystanie przygotowanych danych 
    %         w procesie uczenia i syntezy nowych próbek
    %     \end{itemize}
    % }
}


