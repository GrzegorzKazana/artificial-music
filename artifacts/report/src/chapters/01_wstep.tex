
\chapter{Wstęp}
{
    Współczesne zastosowania metod uczenia maszynowego są bardzo szerokie. Codziennie, często nieświadomie,
    korzystamy z usług i produktów których działanie nie byłoby tak skuteczne lub nawet możliwe bez sztucznej inteligencji.
    Z pośród niezliczonej ilości problemów praktycznie niemożliwych do rozwiązania podejściami klasycznymi, 
    takich jak detekcja i klasyfikacja obiektów na obrazach lub rozpoznawanie mowy,
    duża część zostaje rozwiązana z wykorzystaniem szerokiej gamy algorytmów oraz modeli uczenia.

    Zadania stawiane przed systemami sztucznej inteligencji można podzielić między innymi na problemy:
    \begin{itemize}
        \item klasyfikacji - polegające na przypisaniu przykładu do określonej kategorii, 
        na przykład rozpoznawanie obiektów na obrazie 
        \item grupowania - polegające na wyodrębnieniu przypadków o podobnych cechach 
        lub współwystępujących, na przykład systemy sugerujące przedmioty w sklepach internetowych na podstawie dotychczasowych wyborów
        \item regresji - polegające na oszacowaniu wartości ciągłej dla poszczególnych przykładów, 
        na przykład przewidywanie cen produktów.
    \end{itemize}

    Coraz szersze zastosowanie metod uczenia maszynowego prowadzi do powstawania pytań poruszających temat 
    ich granic możliwości i zastosowań.
    Pomimo powszechności i znaczenia zastosowań przytoczonych kategorii problemów, nie można w nich dostrzec przejawów kreatywności.
    Jednym z problemów, którego rozwiązanie byłoby silnym argumentem w debacie na temat kreatywności i zdolności twórczych
    systemów komputerowych jest komponowanie muzyki.

    \section{Cel i zastosowania}
    {
        Celem pracy jest analiza różnych podejść do obróbki plików muzycznych oraz 
        porównanie wyników sposobów syntezy próbek z wykorzystaniem metod uczenia maszynowego. 

        Pomimo że praca ma charakter badawczo-eksperymentatorski, 
        opracowane metody oraz ich wyniki mogą potencjalnie znaleźć zastosowanie
        w zadaniach ekstrakcji stylu, augmentacji istniejących zbiorów danych,
        lub w optymistycznym scenariuszu jako narzędzie wspierające proces twórczy muzyków.
    }

    \section{Zakres pracy}
    {
        Do zadań autora pracy należy
        \begin{itemize}
            \item dobór zbioru danych
            \item dobór, implementacja i analiza metod przetwarzania plików muzycznych
            \item dobór i parametryzacja architektury modelu uczenia maszynowego
            \item wykorzystanie przygotowanych danych 
            w procesie uczenia i syntezy nowych próbek
            \item ocena uzyskanych wyników
        \end{itemize}
    }


    % Muzyką nazywamy ciągi dźwięków tworzące kompozycyjną całość. 
    % Utwory muzyczne można analizować pod wieloma względami, takimi jak:
    % \begin{itemize}
    %     \item rytmiczność - organizacja dźwięków w czasie
    %     \item melodyczność - sposób zestawiania następujących dźwięków 
    %     \item harmoniczność - spójność i ład występujący między dźwiękami
    %     \item dynamika - zróżnicowanie siły dźwięków
    % \end{itemize}

    % \section{Osadzenie problemu w dziedzinie}
    % {
    %     Generowanie muzyki można rozumieć jako zadanie polegające 
    %     na ekstrakcji pewnych cech charakterystycznych
    %     przykładowych utworów, i wykorzystaniu ich przy syntezie tworzonych próbek. 
    % }

    % \section{Cel pracy}
    % {
    %     Celem pracy jest analiza i porównanie wyników różnych podejść 
    %     do obróbki plików muzycznych oraz sposobów syntezy próbek 
    %     z wykorzystaniem metod uczenia maszynowego. 
    % }

    % \section{Zakres pracy i wkład autora}
    % {
    %     Do zadań autora pracy należy
    %     \begin{itemize}
    %         \item dobór, implementacja, analiza metod przetwarzania plików muzycznych
    %         \item dobór i parametryzacja architektury modelu uczenia maszynowego
    %         \item wykorzystanie przygotowanych danych 
    %         w procesie uczenia i syntezy nowych próbek
    %     \end{itemize}
    % }
}


