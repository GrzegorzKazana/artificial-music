
\chapter{Wstęp}
{
    Współczesne zastosowania metod uczenia maszynowego są bardzo szerokie. Codziennie, często nieświadomie,
    korzystamy z usług i produktów których działanie nie byłoby tak skuteczne lub nawet możliwe bez sztucznej inteligencji.
    Z pośród niezliczonej ilości problemów praktycznie niemożliwych do rozwiązania podejściami klasycznymi, 
    takich jak detekcja i klasyfikacja obiektów na obrazach lub rozpoznawanie mowy,
    duża część zostaje rozwiązana z wykorzystaniem szerokiej gamy algorytmów oraz modeli uczenia.

    Zadania stawiane przed systemami sztucznej inteligencji można podzielić między innymi na problemy:
    \begin{itemize}
        \setlength\itemsep{-0.5em}
        \item klasyfikacji - polegające na przypisaniu obiektu do określonej kategorii, 
        na przykład rozpoznawanie obiektów na obrazie 
        \item grupowania - polegające na wyodrębnieniu przypadków o podobnych cechach 
        lub współwystępujących, na przykład systemy sugerujące przedmioty w sklepach internetowych na podstawie dotychczasowych wyborów
        \item regresji - polegające na oszacowaniu wartości ciągłej dla poszczególnych cech, 
        na przykład przewidywanie cen produktów.
    \end{itemize}

    \pagebreak

    Coraz szersze zastosowanie metod uczenia maszynowego prowadzi do powstawania pytań poruszających temat 
    ich granic możliwości i zastosowań.
    Pomimo powszechności i znaczenia zastosowań przytoczonych kategorii problemów, nie można w nich dostrzec przejawów kreatywności.
    
    Jednym z problemów, którego rozwiązanie byłoby silnym argumentem w debacie na temat kreatywności i zdolności twórczych
    systemów komputerowych jest komponowanie muzyki.

    \section{Cel pracy}
    {
        % Celem pracy jest analiza różnych podejść do obróbki plików muzycznych oraz 
        % porównanie wyników sposobów syntezy próbek z wykorzystaniem metod uczenia maszynowego. 

        Celem pracy jest opracowanie oraz porównanie wyników metod generacji utworów muzycznych za pomocą
        sztucznych sieci neuronowych. Krokiem będącym fundamentem do realizacji celu jest również
        dobór odpowiedniego sposobu przetwarzania plików muzycznych do postaci numerycznej.
        
        % Pomimo że praca ma charakter badawczo-eksperymentalny, 
        % opracowane metody oraz ich wyniki mogą potencjalnie znaleźć zastosowanie
        % w zadaniach ekstrakcji stylu, augmentacji istniejących zbiorów danych,
        % lub w optymistycznym scenariuszu jako narzędzie wspierające proces twórczy muzyków.
        
        Pomyślna realizacja celów i otrzymanie akceptowalnych rezultatów oznaczałoby, że
        metody opracowane w procesie badawczym mają potencjał wdrożeniowy. 
        Zbadane algorytmy oraz kroki mogą potencjalnie znaleźć zastosowanie
        w zadaniach ekstrakcji stylu, augmentacji istniejących zbiorów danych,
        lub w optymistycznym scenariuszu jako narzędzie wspierające proces twórczy muzyków.
    }

    \section{Zakres pracy}
    {
        Do zadań autora pracy należy
        \begin{itemize}
            \setlength\itemsep{-0.5em}
            \item dobór zbioru danych
            \item dobór, implementacja i analiza metod przetwarzania plików muzycznych
            \item dobór i parametryzacja architektury modelu uczenia maszynowego
            \item wykorzystanie przygotowanych danych 
            w procesie uczenia i syntezy nowych próbek
            \item ocena uzyskanych wyników
        \end{itemize}
    }

    \section{Struktura pracy}
    {
        Niniejsza praca w kolejnych rozdziałach prezentuje analizę różnych podejść do kolejnych etapów
        procesu implementacji systemu, którego celem jest generowanie utworów muzycznych. 

        Treścią 2. rozdziału jest przybliżenie dziedziny pracy oraz odniesienie do istniejących prac o powiązanej tematyce.
        
        W rozdziale 3. poruszono problem wyboru formatu danych, oraz przedstawiono opis zbioru na którym
        były przeprowadzane eksperymenty. Przedstawiono również potencjalne trudności wiążące się z wyborem
        poszczególnych źródeł danych.

        Rozdział 4. traktuje o różnych podejściach do zagadnienia reprezentacji wybranego formatu 
        danych w postaci numerycznej. Rozdział zawiera również opis wad i zalet każdego z podejść.

        Treścią 5. rozdziału jest przestawienie obranego typu modelu uczenia maszynowego oraz 
        wybranej architektury, razem z opisem procesu uczenia.

        Rozdział 6. jest rozdziałem poświęconym generacji próbek oraz opisowi przyjętego podejścia oraz sposobów 
        wymuszania procesu generacji.

        Rozdziały 7. i 8. są poświęcone analizie wyników, wnioskom i sugestiom zmian w procesie mających na celu 
        poprawienie otrzymywanych rezultatów.
    }
}


