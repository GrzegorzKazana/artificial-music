
\chapter{Wstęp}
{

%  \begin{itemize}
%      \item wprowadzenie w problem / zagadnienie
%      \item osadzenie problemu w dziedzinie
%      \item cel pracy
%      \item zakres pracy
%      \item zwięzła charakterystyka rozdziałów
%      \item jednoznacznie określenie wkładu autora

%  \end{itemize}

 \section{Wprowadzenie w problem}
 {
    Współczesne zastosowania metod uczenia maszynowego są bardzo szerokie.
    Z pośród niezliczonej ilości problemów praktycznie niemożliwych do rozwiązania podejściami klasycznymi, 
    duża część zostaje rozwiązana z wykorzystaniem szerokiej gamy algorytmów oraz modeli uczenia.
     
    Zadania stawiane przed systemami sztucznej inteligencji można podzielić między innymi na problemy:
    \begin{itemize}
        \item klasyfikacji - polegające na przypisaniu przykładu do określonej kategorii, 
        np. rozpoznawanie obiektów na obrazie 
        \item grupowania - polegające na wyodrębnieniu przypadków o podobnych cechach 
        lub współwystępujących, np. systemy sugerujące przedmioty w sklepach internetowych
        \item regresji - polegające na oszacowaniu wartości ciągłej dla poszczególnych przykładów, 
        np. przewidywanie cen produktów.
    \end{itemize}

    Pomimo ważności oraz znaczenia zastosowań powyższych zagadnień, 
    nie można w nich dostrzec przejawów kreatywności.
    Jednym z problemów, w którym można doszukiwać się inwencji jest tworzenie muzyki.

    % dodać jakieś źródło
    Muzyką według definicji <WSTAW LINK TUTAJ> są ciągi dźwięków tworzące kompozycyjną całość. 
    Utwory muzyczne można analizować pod wieloma względami, takimi jak:
    \begin{itemize}
        \item rytmiczność - organizacja dźwięków w czasie
        \item melodyczność - sposób zestawiania następujących dźwięków 
        \item harmoniczność - spójność i ład występujący między dźwiękami
        \item dynamika - zróżnicowanie siły dźwięków
    \end{itemize}
    W poniższej pracy skupiono się jednak na dwóch aspektach wynikających 
    z powyższej definicji: rytmiczności i tonalności.
 }

 \section{Osadzenie problemu w dziedzinie}
 {
    Generowanie muzyki można rozumieć jako zadanie polegające 
    na ekstrakcji pewnych cech charakterystycznych
    przykładowych utworów, i wykorzystaniu ich przy syntezie tworzonych próbek. 
 }

 \section{Cel pracy}
 {
    Celem pracy jest analiza i porównanie wyników różnych podejść 
    do obróbki plików muzycznych oraz sposobów syntezy próbek 
    z wykorzystaniem metod uczenia maszynowego. 
 }

 \section{Zakres pracy i wkład autora}
 {
    Do zadań autora pracy należy
    \begin{itemize}
        \item dobór, implementacja, analiza metod przetwarzania plików muzycznych
        \item dobór i parametryzacja architektury modelu uczenia maszynowego
        \item wykorzystanie przygotowanych danych 
        w procesie uczenia i syntezy nowych próbek
    \end{itemize}

    
 }
}


