\chapter{Analiza tematu}
{
  Celem zadania doboru i analizy metod obróbki plików muzycznych jest 
  przekształcenie danych do postaci użytecznej przez algorytmy uczenia maszynowego.
  Do istotnych aspektów wyników tego kroku należy między innymi stopień kompresji danych,
  sposób wyrażania relacji między przykładami, możliwość bezstratnej transformacji odwrotnej. 

  Zadanie generowania muzyki polega na ekstrakcji pewnych cech charakterystycznych
  przykładowych utworów, na przykład stylu konkretnego artysty, 
  i wykorzystaniu ich przy syntezie tworzonych próbek. 

  \section{Wprowadzenie do dziedziny}
  {
    Jedną z najbardziej rozpowszechnionych metod uczenia maszynowego jest zastosowanie sieci neuronowych.
    Zaproponowane w latach czterdziestych i rozwijane w drugiej połowie XX wieku \cite{McCulloch1943}, sztuczne sieci neuronowe
    czerpią inspirację z sposobu funkcjonowania ludzkiego mózgu, który jest zbudowany z komórek nerwowych - neuronów.
    Połączenia między komórkami są modelowane poprzez wagi, reprezentujące siłę połączenia, a zjawisku aktywacji
    komórek w sieci odpowiada operacja ważonej sumy informacji pochodzących z połączonych neuronów oraz wag tych połączeń.
    
    W latach siedemdziesiątych opracowano algorytm uczenia sztucznych sieci z powodzeniem wykorzystywany
    do dziś - propagację wsteczną. Metoda ta polega na minimalizacji błędu predykcji poprzez regulację
    wag w oparciu o pochodną funkcji błędu.

    Zastosowania sieci neuronowych są bardzo szerokie i różnorodne. Zasadą będącym podporą uniwersalności 
    sztucznych sieci neuronowych jest twierdzenie stanowiące o ich możliwości aproksymacji dowolnej 
    funkcji ciągłej w zamkniętym przedziale \cite{Csji2001ApproximationWA}.

    Mimo tego, że ta metoda uczenia maszynowego jest znana od wielu dekad, dopiero w ostatnich latach
    przeżywa swoisty renesans spowodowany wzrostem dostępnej mocy obliczeniowej oraz 
    możliwością zastosowania w procesie uczenia akceleratorów (na przykład kart graficznych) znacząco 
    przyśpieszających równoległe operacje matematyczne.


    Ponieważ dane reprezentujące muzykę mają postać sekwencji rozłożonej w czasie, 
    konieczne jest wykorzystanie sieci mających możliwość agregacji stanu, czyli architektur posiadających pamięć.
    Sieciami spełniającymi powyższy warunek są modele należące do grupy rekurencyjnych sieci neuronowych.
    Najprostszym przykładem rekurencyjnej sieci neuronowej jest sieć na której wejście 
    przekazywany jest również stan wyjść z analizy poprzedniego elementu sekwencji.
    
    Niestety, taka architektura jest narażona na wiele problemów, takich jak trudność
    tworzenia powiązań pomiędzy odległymi elementami sekwencji oraz zjawisko znikającego lub eksplodującego gradientu.

    Architekturą rozwiązującą powyższe problemy jest architektura
    Long Short-term Memory network (LSTM) \cite{LongShortTermMemory}. 
    Jej pomysłodawca rozszerza klasyczną rekurencyjną sieć neuronową o globalną pamięć niezależną
    od stanu wyjść. Dane mogą zostać wprowadzone lub wymazane z pamięci dzięki warstwom bramkującym. 
    Kosztem większych możliwości architektury jest zwielokrotnienie 
    ilości parametrów, co może przekładać się na dłuższy czas uczenia.
  }

  \newpage

  \section{Założenia}
  {
    Na potrzeby pracy została przyjęta definicja muzyki stanowiąca, że muzyką nazywamy ciągi 
    dźwięków tworzące kompozycyjną całość. 

    Utwory muzyczne można analizować pod wieloma względami, takimi jak:
    \begin{itemize}
        \item rytmiczność - organizacja dźwięków w czasie
        \item melodyczność - sposób zestawiania następujących dźwięków 
        \item harmoniczność - spójność i ład występujący między dźwiękami
        \item dynamika - zróżnicowanie siły dźwięków
    \end{itemize}

    W kontekście pracy skupiono się na dwóch aspektach wynikających z powyższej 
    definicji: rytmiczności i tonalności.
  }

  \section{Przegląd literatury}
  {
    W podobnej tematyce powstało wiele prac, lecz każda wyróżnia 
    się innym podejściem do problemu.

    Elementem wspólnym dużej części prac jest wykorzystany format danych wejściowych.
    W większości studiowanych artykułów, twórcy decydują się na wykorzystanie 
    danych muzycznych w formacie midi \cite{Yang2017MidiNetAC, Hadjeres2016DeepBachAS, Wu2017AHR, Mogren2016CRNNGANCR}, 
    aczkolwiek równie popularnym wyjściem jest skorzystanie z plików notacji ABC \cite{Agarwala2017MusicCU, Sturm2015FolkMS}. 
    Najrzadsze jest podejście opierające się o wykorzystanie nieprzetworzonych plików audio, 
    na przykład w formacie wave \cite{Oord2016WaveNetAG}.
    
    Jedną z różnic występujących między podejściami jest sposób obróbki i tworzenia
    wewnętrznej reprezentacji danych, gdyż praktycznie wszyscy autorzy 
    proponują własne rozwiązanie problemu. 
    W przypadku danych tekstowych w formacie ABC, częstym elementem jest ograniczenie
    lub usunięcie meta informacji zawartych w plikach \cite{Agarwala2017MusicCU, Sturm2015FolkMS} oraz pominięcie białych znaków.
    Konkretne znaki zapisu są albo przedstawione jako wektory kodu 1 z N \cite{Sturm2015FolkMS}, 
    lub przekształcane są kolejne wartości liczb naturalnych \cite{Agarwala2017MusicCU}.
    W pracach korzystających z plików midi, częstym podejściem do problemu reprezentacji wysokości
    dźwięku jest zastosowanie kodu 1 z N \cite{Yang2017MidiNetAC}. Większym zróżnicowaniem cechuje się podejście
    do reprezentacji danych w wymiarze czasu. Duża część podejść opiera się na restrykcji możliwych
    wartości rytmicznych do ustalonego dyskretnego zbioru \cite{Hadjeres2016DeepBachAS, Wu2017AHR}, 
    podczas gdy stosowane są również podejścia korzystające bezpośrednio z taktowania 
    plików midi \cite{Mogren2016CRNNGANCR}.

    Kluczowym aspektem wszystkich prac jest dobór modelu uczenia maszynowego.
    Najbardziej popularnym wyborem są rekurencyjne sieci neuronowe \cite{Agarwala2017MusicCU, Sturm2015FolkMS, Souza2018DeepNN, Mogren2016CRNNGANCR},
    ale opisywane są również inne podejścia, na przykład:
    \begin{itemize}
      \item modele kontekstowe, takie jak Continous Bag-of-Words \cite{Agarwala2017MusicCU},
      \item modele generatywno-adwersyjne (Generative Adversial Networks) \cite{Agarwala2017MusicCU, Mogren2016CRNNGANCR}
      \item modele oparte o sieci konwolucyjne \cite{Yang2017MidiNetAC, Oord2016WaveNetAG}.
    \end{itemize}

    Aspektem wartym zwrócenia uwagi, są również poczynione założenia i nałożone
    ograniczenia. Niektóre metody przytoczone przez autorów prac wymagają:
    \begin{itemize}
      \item pominięcia elementów polifonicznych utworów \cite{Sturm2015FolkMS, Hadjeres2016DeepBachAS},
      \item ograniczenia zbioru danych do utworów będących w 
      ustalonej tonacji i/lub ustalonym metrum \cite{Sturm2015FolkMS, Wu2017AHR},
      \item ograniczenia występujących wartości rytmicznych \cite{Hadjeres2016DeepBachAS}.
    \end{itemize}
  }

  \section{Odniesienie do istniejących prac}
  {
    Niniejsza praca w kolejnych rozdziałach prezentuje analizę różnych podejść do kolejnych etapów
    procesu implementacji systemu, którego celem jest generowanie utworów muzycznych. 

    W rozdziale 3. poruszono problem wyboru formatu danych, oraz przedstawiono opis zbioru na którym
    były przeprowadzane eksperymenty. Przedstawiono również potencjalne trudności wiążące się z wyborem
    poszczególnych źródeł danych.

    W kolejnym rozdziale omówiono różne podejścia do zagadnienia reprezentacji wybranego formatu 
    danych w postaci numerycznej. Rozdział zawiera również opis wad i zalet każdego z podejść.

    Treścią piątego rozdziału jest przestawienie obranego typu modelu uczenia maszynowego oraz 
    wybranej architektury, razem z opisem procesu uczenia.

    Rozdział poświęcony generacji próbek zawiera opis przyjętego podejścia oraz sposobów 
    wymuszania procesu generacji.

    Dalsze rozdziały poświęcone są analizie wyników, wnioskom i sugestiom zmian w procesie mających na celu 
    poprawienie otrzymywanych rezultatów.
  }

  \section{Opis narzędzi}
  {
    Podczas implementacji korzystano z narzędzi:
    \begin{itemize}
      \item Google Colab - usługa będąca środowiskiem obliczeniowym przystosowanym 
      do uruchamiania skryptów w języku Python, z dostępem do akceleratorów obliczeniowych
      w postaci kart graficznych i jednostek tensorowych,
      \item Tensorflow - biblioteka języka Python służąca do wydajnych obliczeń, ułatiwająca 
      tworzenie i uczenia głębokich sieci neuronowych. Podczas pracy skorzystano z 
      wysokopoziomowego interfejsu tf.keras, zawierającego predefiniowane architektury i warstwy sieci,
      \item Jupyter notebook - środowisko uruchomieniowe języka Python, umożliwiające przejrzystą
      ilustrację wykonywanych operacji,
      \item mido - biblioteka ułatwiająca otwieranie i przetwarzania plików w formacie midi.
    \end{itemize}
  }
}