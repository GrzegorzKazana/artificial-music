\chapter{Analiza tematu}
{
  % \begin{itemize}
  %   \item wprowadzenie do dziedziny (state of art)
  %   \item sformułowanie problemu
  %   \item studia literaturowe / przegląd literatury tematu
  %   \item opis znanych rozwiązań (także opisanych naukowo)
  %   \item opis algorytmów (obcych)
  %   \item osadzenie pracy w kontekście
  % \end{itemize}

  Celem zadania doboru i analizy metod obróbki plików muzycznych jest 
  przekształcenie danych do postaci użytecznej przez algorytmy uczenia maszynowego.
  Do istotnych aspektów tego kroku należy między innymi stopień kompresji danych,
  sposób wyrażania relacji między przykładami, możliwość bezstratnej transformacji odwrotnej. 

  Zadanie generowania muzyki polega na ekstrakcji pewnych cech charakterystycznych
  przykładowych utworów, np. stylu konkretnego artysty, 
  i wykorzystaniu ich przy syntezie tworzonych próbek. 

  \section{Wprowadzenie do dziedziny}
  {
    Jedną z najbardziej rozpowszechnionych metod uczenia maszynowego jest zastosowanie sieci neuronowych.
    %%% tu można by cyknąć link do artykułu %%%
    Zaproponowane w latach czterdziestych i rozwijane w drugiej połowie XX wieku, sztuczne sieci neuronowe
    czerpią inspirację z sposobu funkcjonowania ludzkiego mózgu zbudowanego z komórek nerwowych - neuronów.
    Połączenia między komórkami są modelowane poprzez wagi, reprezentujące siłę połączenia, a zjawisko aktywacji
    komórek w sieci poprzez operację ważonej sumy informacji z połączonych neuronów oraz ich wag.
    %%% tu też można odwołanie %%%
    W latach siedemdziesiątych opracowano algorytm uczenia sztucznych sieci z powodzeniem wykorzystywany
    do dziś - propagację wsteczną. Metoda ta polega na minimalizacji błędu predykcji poprzez regulację
    wag w oparciu o pochodną funkcji błędu.
    %%% odnośnik do czegoś z 'Universal approximation theorem'
    Zastosowania sieci neuronowych są szerokie, co potwierdza twierdzenie stanowiące 
    o ich możliwości aproksymacji dowolnej funkcji ciągłej w zamkniętym przedziale.

    Mimo tego, że ta metoda uczenia maszynowego jest znana od wielu dekad, dopiero w ostatnich latach
    przeżywa swoisty renesans spowodowany wzrostem dostępnej mocy obliczeniowej oraz 
    możliwością zastosowania w procesie uczenia kart graficznych znacząco 
    przyśpieszających równoległe operacje matematyczne.


    Ponieważ dane reprezentujące muzykę mają postać sekwencji rozłożonej w czasie, 
    konieczne jest wykorzystanie sieci mających możliwość agregacji stanu tj. posiadające pamięć.
    Sieciami spełniającymi powyższy warunek są modele należące do grupy rekurencyjnych sieci neuronowych.
    Najprostszym przykładem rekurencyjnej sieci neuronowej jest sieć na której wejście 
    przekazywany jest również stan wyjść z analizy poprzedniego elementu sekwencji.
    %%% obrazek RNN ???
    Niestety, taka architektura jest narażona na wiele problemów, takich jak trudność
    tworzenia powiązań pomiędzy odległymi elementami sekwencji oraz znikający gradient.

    %%% link do artykułu o lstmach %%%
    Architekturą rozwiązującą powyższe problemy jest zasugerowana przez XXXXXX architektura
    Long short-term memory network (LSTM). Kosztem jej większych możliwości jest zwielokrotnienie 
    ilości parametrów, co może przekładać się na dłuższy czas uczenia.
    %%% obrazek  lstm ???
  }

  \section{Założenia}
  {
    Na potrzeby pracy została przyjęta następująca definicja muzyki:

    Muzyką nazywamy ciągi dźwięków tworzące kompozycyjną całość. 

    Utwory muzyczne można analizować pod wieloma względami, takimi jak:
    \begin{itemize}
        \item rytmiczność - organizacja dźwięków w czasie
        \item melodyczność - sposób zestawiania następujących dźwięków 
        \item harmoniczność - spójność i ład występujący między dźwiękami
        \item dynamika - zróżnicowanie siły dźwięków
    \end{itemize}

    W kontekście pracy skupiono się na dwóch aspektach wynikających z powyższej 
    definicji: rytmiczności i tonalności.
  }

  \section{Przegląd literatury}
  {

  }

  \section{Odniesienie do istniejących prac}
  {

  }

  \section{Opis narzędzi}

  % \section{Sformułowanie problemu}
  % {
  %   W celu realizacji przedstawionego tematu, konieczna będzie praca w etapach
  %   doboru i obróbki danych wejściowych, oraz projektu i wykorzystania modelu uczenia maszynowego.
  % }
}