\chapter{Generowanie próbek} 
{
    Opierając się na założeniu, że wyuczony model posiada pewną wiedzę możliwe jest przejście
    do procesu generowania próbek.

    \section{Opis podejścia}
    {
        Proces otrzymywania generowanych sekwencji zależny jest od rodzaju modelu.
        W przypadku modeli typu sequence-to-sequence, możliwe byłoby podanie losowego 
        wektora na część modelu w którym dokonuje się przekształcenie jednowymiarowej postaci ukrytej na sekwencję.
        Przy zastosowaniu architektury opisanej w poprzednim rozdziale, konieczne będzie 
        rozpoczynanie generacji sekwencją, dalej nazywaną ziarnem. 

        Ponieważ opracowany model na przekształca sekwencje danych wejściowych na sekwencje
        o tej samej długości, sekwencja otrzymana po zadaniu ziarna będzie tej samej długości.
        Proponowanym rozwiązaniem tego problemu jest łączenie sekwencji wejściowych z sekwencjami wyjściowymi,
        i ponowne wprowadzenie jej do sieci. Krok ten jest powtarzany aż do otrzymania sekwencji o pożądanej długości.
    }

    \section{Opisy ziaren}
    {
        Kolejną kwestią, którą rozważono, jest opracowanie sposobów generowania ziaren.
        Oczywistym pomysłem jest zastosowanie wektorów o wartościach zerowych lub losowych.

        Ponadto, opracowano ziarna zawierające pojedynczy dźwięk, sekwencję losowych dźwięków
        i sekwencję dźwięków często współwystępujących. Dźwięki współwystępujące są obierane poprzez
        porównanie odległości między składowymi ich wektorów i wybranie wektorów o zbliżonych składowych.
    }
}